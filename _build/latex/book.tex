%% Generated by Sphinx.
\def\sphinxdocclass{jupyterBook}
\documentclass[letterpaper,10pt,english]{jupyterBook}
\ifdefined\pdfpxdimen
   \let\sphinxpxdimen\pdfpxdimen\else\newdimen\sphinxpxdimen
\fi \sphinxpxdimen=.75bp\relax
\ifdefined\pdfimageresolution
    \pdfimageresolution= \numexpr \dimexpr1in\relax/\sphinxpxdimen\relax
\fi
%% let collapsible pdf bookmarks panel have high depth per default
\PassOptionsToPackage{bookmarksdepth=5}{hyperref}
%% turn off hyperref patch of \index as sphinx.xdy xindy module takes care of
%% suitable \hyperpage mark-up, working around hyperref-xindy incompatibility
\PassOptionsToPackage{hyperindex=false}{hyperref}
%% memoir class requires extra handling
\makeatletter\@ifclassloaded{memoir}
{\ifdefined\memhyperindexfalse\memhyperindexfalse\fi}{}\makeatother

\PassOptionsToPackage{booktabs}{sphinx}
\PassOptionsToPackage{colorrows}{sphinx}

\PassOptionsToPackage{warn}{textcomp}

\catcode`^^^^00a0\active\protected\def^^^^00a0{\leavevmode\nobreak\ }
\usepackage{cmap}
\usepackage{fontspec}
\defaultfontfeatures[\rmfamily,\sffamily,\ttfamily]{}
\usepackage{amsmath,amssymb,amstext}
\usepackage{polyglossia}
\setmainlanguage{english}



\setmainfont{FreeSerif}[
  Extension      = .otf,
  UprightFont    = *,
  ItalicFont     = *Italic,
  BoldFont       = *Bold,
  BoldItalicFont = *BoldItalic
]
\setsansfont{FreeSans}[
  Extension      = .otf,
  UprightFont    = *,
  ItalicFont     = *Oblique,
  BoldFont       = *Bold,
  BoldItalicFont = *BoldOblique,
]
\setmonofont{FreeMono}[
  Extension      = .otf,
  UprightFont    = *,
  ItalicFont     = *Oblique,
  BoldFont       = *Bold,
  BoldItalicFont = *BoldOblique,
]



\usepackage[Bjarne]{fncychap}
\usepackage[,numfigreset=1,mathnumfig]{sphinx}

\fvset{fontsize=\small}
\usepackage{geometry}


% Include hyperref last.
\usepackage{hyperref}
% Fix anchor placement for figures with captions.
\usepackage{hypcap}% it must be loaded after hyperref.
% Set up styles of URL: it should be placed after hyperref.
\urlstyle{same}


\usepackage{sphinxmessages}



        % Start of preamble defined in sphinx-jupyterbook-latex %
         \usepackage[Latin,Greek]{ucharclasses}
        \usepackage{unicode-math}
        % fixing title of the toc
        \addto\captionsenglish{\renewcommand{\contentsname}{Contents}}
        \hypersetup{
            pdfencoding=auto,
            psdextra
        }
        % End of preamble defined in sphinx-jupyterbook-latex %
        

\title{My sample book}
\date{Jan 04, 2025}
\release{}
\author{The Jupyter Book Community}
\newcommand{\sphinxlogo}{\vbox{}}
\renewcommand{\releasename}{}
\makeindex
\begin{document}

\pagestyle{empty}
\sphinxmaketitle
\pagestyle{plain}
\sphinxtableofcontents
\pagestyle{normal}
\phantomsection\label{\detokenize{intro::doc}}



\begin{itemize}
\item {} 
\sphinxAtStartPar
{\hyperref[\detokenize{modelo::doc}]{\sphinxcrossref{Modelo matemático para o Pêndulo Interrompido}}}

\end{itemize}

\sphinxstepscope


\chapter{Modelo matemático para o Pêndulo Interrompido}
\label{\detokenize{modelo:modelo-matematico-para-o-pendulo-interrompido}}\label{\detokenize{modelo::doc}}
\sphinxAtStartPar
Vamos analisar um modelo matemático para descrever a trajetória de uma massa num \sphinxstyleemphasis{loop} ou para um pêndulo interrompido. Vamos admitir que, dependendo da velocidade inicial e energia cinética na parte mais baixo do círculo, a trajetória pode
\begin{enumerate}
\sphinxsetlistlabels{\arabic}{enumi}{enumii}{}{.}%
\item {} 
\sphinxAtStartPar
ser parcialmente circular e periódica: a velocidade inicial não é suficiente para subir mais do que um raio do círculo;

\item {} 
\sphinxAtStartPar
ser inicialmente circular, mas a partir de uma certa altura ou ângulo crítico a massa se “desprende” da trajetória circular, a força normal ou da tensão é zerada e a massa segue em queda livre;

\item {} 
\sphinxAtStartPar
ser completamente circular: a velocidade incicial é suficiente para completar o círculo.

\end{enumerate}

\sphinxAtStartPar
Durante a parte circular da trajetória circular há forças de vínculo (a força normal e a tensão no fio respectivamente) fornecendo parte da força centrípeta. Em ambos os casos há a força da gravidade, com componentes perpendicular (fornecendo uma parte da força centrípeta) e tangencial. Há, obviamente, também forças de atrito e graus de liberdade (rotação da massa, movimento do fio) que não vamos considerar inicialmente.

\sphinxAtStartPar
Nossa referência é o centro do círculo (0,0). Vamos chamar o raio do círculo \(h\) e soltamos a massa \(m\) de uma altura \(H\). Em nosso referencial, se o ponto mais baixo da trajetória é \(-h\) e se a gente solta a massa de \(H \leq 0\), estamos na situação 1 acima, a de movimento periódico.  Se \(H \geq 0\), a velocidade \(v_0\) em \((-h,0)\) é dado por
\begin{equation}\label{equation:modelo:eq1}
\begin{split}
\frac{1}{2}v_{0}^{2} = gH 
\end{split}
\end{equation}
\sphinxAtStartPar
A equação \eqref{equation:modelo:eq1} assume que soltamos a massa com velocidade zero.


\section{Condição para \sphinxstyleemphasis{justamente} completar o círculo}
\label{\detokenize{modelo:condicao-para-justamente-completar-o-circulo}}
\sphinxAtStartPar
É instrutivo começar com o caso em que a massa atinge \sphinxstyleemphasis{justamente} a altura \(h\) (o terceiro caso das três possibilidades descritas acima). Qual é a altura que precisamos solter a massa para isso acontecer? Parece o tipo de problema onde é suficiente usar considerações de conservação de energia mecânica. Uma análise ingênua, de um iniciante, poderia concluir que precisamos soltar a massa de \(H=h\). Afinal, meste caso, a energia potencial inicial \(mgH\) é convertida totalmente para a energia potencial final \(mgh\). A massa não “quer” chegar na mesma altura de onde partiu?

\sphinxAtStartPar
Esse raciocício ignora o fato que a velocidade da massa no topo da trajetória (vamos chamar de \(v_t\)) não pode ser zero. Para nosso modelo, a força de vínculo (a força normal ou a tensão no fio) pode ser zero no topo da trajetória mas para continuar em movimento circular deve haver uma a força centrípeta, dado somente pela gravidade, resultando numa velocidade mínima no topo
\begin{equation*}
\begin{split} \frac{v_t^2}{h} = g \end{split}
\end{equation*}
\sphinxAtStartPar
Combinando com a conservação de energia mecânica
\begin{equation*}
\begin{split} \frac{1}{2}v_{t}^2 = \frac{1}{2}v_0^2 - gh \end{split}
\end{equation*}
\sphinxAtStartPar
e temos que
\begin{equation*}
\begin{split} \frac{1}{2}gh + gh = gH \Rightarrow H = \frac{3}{2} h \end{split}
\end{equation*}
\sphinxAtStartPar
Ou seja, precisamos soltar a massa de uma altura um pouco \sphinxstyleemphasis{mais} do que \(h\) para \sphinxstyleemphasis{justamente} chegar na altura \(h\) no topo da trajetória, porque no nosso modelo no topo há uma aceleração centrípeta mínima (quando a força normal ou de tensão no fio são zero), dado pela aceleração da gravidade \(g\).


\section{Condição para \sphinxstyleemphasis{justamente} passar pelo centro do círculo}
\label{\detokenize{modelo:condicao-para-justamente-passar-pelo-centro-do-circulo}}
\sphinxAtStartPar
Se soltar a massa de uma altura \(0 < H <  \frac{3}{2}h \) em algum momento antes de chegar no topo a magnitude da força de vínculo se torna zero e a massa segue em queda livre. Começando com velocidade inicial \(v_0\) em \((-h,0)\), perguntamos para qual ângulo \(\theta_c\) a força de vínculo (a tensão ou força normal) se torna zero.

\sphinxAtStartPar
XXX inserir uma figura

\sphinxAtStartPar
No momento crítico em que a força de vínculo é zero, a força centrípeta é dado (somente) pela componente perpendicular da força gravitação e temos
\begin{equation}\label{equation:modelo:eq2}
\begin{split}
\frac{v_{c}^{2}}{h}  = g \sin{\theta_c}.
\end{split}
\end{equation}
\sphinxAtStartPar
Por outro lado, pela conservação de energia mecânica
\begin{equation}\label{equation:modelo:eq3}
\begin{split}
  \frac{1}{2}v_{c}^2 = \frac{1}{2}v_{0}^2 -gh\sin{\theta_c}
\end{split}
\end{equation}
\sphinxAtStartPar
Combinando \eqref{equation:modelo:eq2} e \eqref{equation:modelo:eq3}, e usando \(H_c\) (para a “altura de soltura crítica”) em \eqref{equation:modelo:eq1}, temos
\begin{equation}\label{equation:modelo:eq4}
\begin{split}
  \sin{\theta_c} = \frac{2}{3}\frac{H_c}{h}.
\end{split}
\end{equation}
\sphinxAtStartPar
Agora vamos escrever as equações horárias para a parábola de queda livre \(x(t)\) e \(y(t)\) com a posição inicial \((-h\cos{\theta_c},h\sin{\theta_c})\) e velocidade inicial \(\vec{v_c}\):
\begin{equation*}
\begin{split}
\begin{aligned}
x(t) & = -h\cos{\theta_c} + v_c\sin{\theta_c}t  \\
y(t) & = h\sin{\theta_c} + v_c\cos{\theta_c}t - \frac{1}{2}gt^2
\end{aligned}
\end{split}
\end{equation*}
\sphinxAtStartPar
Para que essa parábola passa pelo \((0,0)\) podemos mostrar que precisamos ter \(\sin{\theta_c} = \sqrt{1/3}\) (ou \(\theta_c = 35.3^{\circ}\)). Inserindo isso em \eqref{equation:modelo:eq4} chegamos a
\begin{equation*}
\begin{split}
  \boxed{\frac{H_c}{h} = \frac{\sqrt{3}}{2}}
\end{split}
\end{equation*}
\sphinxAtStartPar
Ou seja, precisamos soltar a massa de uma altura de \(0.866\) do raio para que a parábola exatamente passa pelo orígem do círculo. Um pouco mais, e o fio se enrola em volta do pino no caso de um pêndulo interrompido. Note que essa altura de soltura é um pouco menos do que o Galileu afirmou (ele disse que seria necessário usar uma altura \(h\) ou exato um raio).

\sphinxAtStartPar
Para terminar, vamos mostrar que \(\sin{\theta_c} = \sqrt{1/3}\). Certamente deve de ter uma maneira geométrica, pelas propriedades de círculos e parábolas de mostrar isso. Mas fiz por força bruta, colocando \(x=0, y=0\) nas equações horárias do parábola. Usando \(x(t_0) = 0\) temos
\begin{equation*}
\begin{split}
t_0 = \frac{h\cos{\theta_c}}{v_c}
\end{split}
\end{equation*}
\sphinxAtStartPar
Inserindo este tempo em \(y(t_0) = 0\) temos
\begin{equation*}
\begin{split}
\frac{1}{2}g \frac{h^2}\cos{\theta_c}{v_{c}^2\sin{\theta_c}^2} = h\sin{\theta_c} + v_c\cos{\theta_c}\frac{h\cos{\theta_c}}{v_{c}\sin{\theta_c}}
\end{split}
\end{equation*}
\sphinxAtStartPar
Pulando alguns passos e usando \(v_{c}^2 = hg\sin{\theta_c}\) chegamos a
\begin{equation*}
\begin{split}
\frac{\sin{\theta_c}²}{\cos{\theta_c}^2} = \frac{1}{2}
\end{split}
\end{equation*}
\sphinxAtStartPar
o que mostra que \(\sin{\theta_c} = \sqrt{1/3}\) (e \(\cos{\theta_c} = \sqrt{2/3}\)).







\renewcommand{\indexname}{Index}
\printindex
\end{document}